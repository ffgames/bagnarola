\section{Valutazioni Finali e Conclusioni}
Al termine della fase di sviluppo abbiamo condotto diversi test al fine
di valutare l'esperienza di gioco.
In questa fase abbiamo appreso appieno l'importanza dell'utilizzo dei
design pattern e dell'accurata strutturazione gerarchica del modello delle
classi. In particolare il pattern MVC ci ha permesso di correggere
facilmente gli errori di implementazione.

Ci siamo concentrati sulla comprensione della correttezza del comportamento delle comunicazioni
tra i nodi della rete.

Dopo gli ultimi bugfix, non abbiamo più riscontrato problemi implementativi
nella fase finale di test e le interazioni fra i giocatori funzionano
come da aspettative. Inoltre l'esperienza di gioco risulta essere
particolarmente fluida e piacevole grazie alle ottimizzazioni introdotte
nell'interfaccia utente.

Per quanto riguarda l'affidabilità, la nostra implementazione è
tollerante ai guasti di tipo crash. La rete logica in Bajnarola infatti si
riorganizza automaticamente nel caso in cui uno o più nodi subiscano un
crash, permettendo agli utenti ancora in gioco di concludere la partita ed essere notificati dei guasti.\\

In conclusione tramite questo lavoro abbiamo avuto l'occasione di
approfondire il mondo dei sistemi distribuiti e allo stesso tempo
siamo riusciti a raggiungere un risultato implementativo molto
soddisfacente che verrà sicuramente arricchito da sviluppi futuri.
